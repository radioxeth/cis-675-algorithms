\documentclass[12pt, letterpaper, twoside]{article}
\usepackage[utf8]{inputenc}
\usepackage{listings}
\usepackage{color}
\usepackage{tikz,pgfplots}

\definecolor{dkgreen}{rgb}{0,0.6,0}
\definecolor{gray}{rgb}{0.5,0.5,0.5}
\definecolor{mauve}{rgb}{0.58,0,0.82}

\lstset{frame=tb,
  language=C,
  aboveskip=2mm,
  belowskip=2mm,
  showstringspaces=false,
  columns=flexible,
  basicstyle={\small\ttfamily},
  numbers=none,
  numberstyle=\tiny\color{gray},
  keywordstyle=\color{blue},
  commentstyle=\color{dkgreen},
  stringstyle=\color{mauve},
  breaklines=true,
  breakatwhitespace=true,
  tabsize=2
}
\title{%
Design and Analysis of Algorithms\\
\large Graphs 3.2.8
}
\author{Daniel Shannon}
\date{April 20th, 2022}

\begin{document}

\begin{titlepage}
\maketitle
\end{titlepage}
\begin{quote}
    Suppose you are given a graph G in adjacency matrix form.

    Goal: Given a vertex u, output all other vertices that are reachable from u (in the same component).    
\end{quote}
You could store a hashtable of the nodes and traverse the graph in a depth first manner. Each time you come to a node, add it to the hash table.

\begin{lstlisting}
void find_nodes(vertex U, graph G)                  
{
    stack = new stack()
    while (stack is not empty){
        traverse node list push undiscovered nodes to stack
        search each node for more undiscovered nodes recursively
        pop from stack and print the node if it hasn't been marked as discovered in the hash table.
    }
    return;
}
\end{lstlisting}
\end{document}
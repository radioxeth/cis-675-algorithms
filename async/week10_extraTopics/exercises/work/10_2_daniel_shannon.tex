\documentclass[12pt, letterpaper, twoside]{article}

\usepackage[utf8]{inputenc}
\usepackage{listings}
\usepackage{color}
\usepackage{algorithm}
\usepackage{algpseudocode}
\usepackage{graphicx}
\usepackage{amsmath}
\usepackage{algpseudocode}
\usepackage{enumitem}

\graphicspath{ {./images/} }

\definecolor{dkgreen}{rgb}{0,0.6,0}
\definecolor{gray}{rgb}{0.5,0.5,0.5}
\definecolor{mauve}{rgb}{0.58,0,0.82}

\lstset{frame=tb,
  language=C,
  aboveskip=2mm,
  belowskip=2mm,
  showstringspaces=false,
  columns=flexible,
  basicstyle={\small\ttfamily},
  numbers=none,
  numberstyle=\tiny\color{gray},
  keywordstyle=\color{blue},
  commentstyle=\color{dkgreen},
  stringstyle=\color{mauve},
  breaklines=true,
  breakatwhitespace=true,
  tabsize=2
}

\title{%
Design and Analysis of Algorithms\\
\large 10.2 Useful to Your Work
}
\author{Daniel Shannon}
\date{June 10th, 2022}

\begin{document}
\begin{titlepage}
\maketitle
\end{titlepage}
\section*{10.2}
\begin{quote}
    please submit examples of how concepts from this class might be useful to problems you have seen in your work or in other projects
\end{quote}

One of my favorite parts of this class has been all of the "ah ha" moments when making connections with my work.
I am currently a senior software engineer at Enverus, which is an Enegery data SaaS company. I have built many applications that 
allow users to derive information from data with the algorithm abstracted away from them.

One suite of software I helped build was OptiFlo. OptiFlo is a network flow problem! The idea is that in North America,
there are Supply and Demand regions of oil and gas, and there are limited ways to move these crude products around.
How do we best move and distribute oil from where it is being produced to where it is needed. There are many constraints
on the transportation. For instance, gas can only be moved via pipelines, and those pipelines have \textbf{capacity} and costs
associated with them known as tarrifs, which are additional \textbf{constraints}. I did not program the algorithm, which uses an engine called GAMS,
but I did build the UI and service that allows users to update constraints and visualize the network flow :D.

More recently, I have been working in a distributed database called Kinetica. Kinetica is a cluster of machines that shard and distribute
datasets across a DAG. I'm not too familiar with Kinetica as it is not my primary job responsibility, but I often hear of certain joins
bringing data to the head node, and how other queries can cause undesireable resharding and projections. Knowing that Kinetica is likely a graph, or DAG in nature,
I have a better appreciation for why these things \emph{might} be happening and how to think about the datasets we put into the database
knowing the constraints of DAGs and how to leverage them.

\end{document}

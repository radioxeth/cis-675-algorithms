\documentclass[12pt, letterpaper, twoside]{article}

\usepackage[utf8]{inputenc}
\usepackage{listings}
\usepackage{color}
\usepackage{algorithm}
\usepackage{algpseudocode}
\usepackage{graphicx}
\usepackage{amsmath}
\usepackage{algpseudocode}
\usepackage{enumitem}

\graphicspath{ {./images/} }

\definecolor{dkgreen}{rgb}{0,0.6,0}
\definecolor{gray}{rgb}{0.5,0.5,0.5}
\definecolor{mauve}{rgb}{0.58,0,0.82}

\lstset{frame=tb,
  language=C,
  aboveskip=2mm,
  belowskip=2mm,
  showstringspaces=false,
  columns=flexible,
  basicstyle={\small\ttfamily},
  numbers=none,
  numberstyle=\tiny\color{gray},
  keywordstyle=\color{blue},
  commentstyle=\color{dkgreen},
  stringstyle=\color{mauve},
  breaklines=true,
  breakatwhitespace=true,
  tabsize=2
}

\title{%
Design and Analysis of Algorithms\\
\large 6.3 Dynamic Programming Exercises
}
\author{Daniel Shannon}
\date{May 18th, 2022}

\begin{document}
\begin{titlepage}
\maketitle
\end{titlepage}
\section*{7.4.2}
\begin{quote}
    Suppose you have a binary array.

    You are using it as a counter:
    \begin{itemize}
        \item In each increment, increase value by 1.
        \item e.g. $[0,0,1,1,1,0,1]\rightarrow[0,0,1,1,1,1,0]$
    \end{itemize}
    Suppose that it costs \$1 to flip a bit. 
    
    Keep a "bank" for each bit.
    
    Suppose whenever you flip a bit from $0\rightarrow1$, you pay \$2: \$1 to flip, \$1 to that bit's bank.

    Perform amortized runing-time analysis.

    Start at right. If 0, flip to 1, terminate. If 1, flip to 0, move left and repeat.
    
\end{quote}
\begin{enumerate}
    \item How many 0$\rightarrow$1 flips per increment?
    
    1 flip per increment
    \item What is the cost per increment?
    
    \$2 per increment
    \item For a 1$\rightarrow$1 flip, how much money is in the bit's bank?
    
    \$1 will be in the bit's bank for a 1$\rightarrow$0 flip.
\end{enumerate}

Amoratized cost \$2 $O(n)$

\newpage
\section*{7.4.4}
\begin{quote}
    Atomic ops=individual flips
\end{quote}
\begin{itemize}
    \item Over $n$ increments, how often do we flip the first bit?
    
    $2^n$ times

    Answer: \textbf{n}
    \item How often do we flip the second bit?
    
    $2^{n-1}$

    Answer: \textbf{$\frac{n}{2}$}
    \item How often do we flip the $k^{th}$ bit?
    
    $2^{n-k}$
    
    Answer: \textbf{$\frac{n}{2^{k-1}}$}
    \item What is the total running time?
    
    $log(n)$

    Answer: \textbf{$n+\frac{n}{2}+\frac{n}{4}+...\le{2n}$}
    $O(n)$
\end{itemize}
\newpage
\section*{7.4.6}
\begin{itemize}
    \item Suppose you have a different binary counter.
    \item But now the cost to flip the $k^{th}$ bit is $2^k$ ($k$ starts at 0).
    \item What is the total running time?
    \begin{itemize}
        \item Over $n$ increments, how often do we flip the first bit?
    
        $2^n$ times

        Answer: \textbf{n}
        \item How often do we flip the second bit?
        
        $2^{n/2}$

        Answer: \textbf{n/2}
        \item How often do we flip the $k^{th}$ bit?
        
        $2^{n-k}$
        \item What is the total running time?
        
        $log(n)$

        Answer: \textbf{$n+2\frac{n}{2}+4\frac{n}{4}+...=n+n+n...=O(nlog(n))$}
    \end{itemize}
\end{itemize}
\end{document}
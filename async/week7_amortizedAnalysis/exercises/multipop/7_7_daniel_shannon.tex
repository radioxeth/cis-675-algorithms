\documentclass[12pt, letterpaper, twoside]{article}

\usepackage[utf8]{inputenc}
\usepackage{listings}
\usepackage{color}
\usepackage{algorithm}
\usepackage{algpseudocode}
\usepackage{graphicx}
\usepackage{amsmath}
\usepackage{algpseudocode}
\usepackage{enumitem}

\graphicspath{ {./images/} }

\definecolor{dkgreen}{rgb}{0,0.6,0}
\definecolor{gray}{rgb}{0.5,0.5,0.5}
\definecolor{mauve}{rgb}{0.58,0,0.82}

\lstset{frame=tb,
  language=C,
  aboveskip=2mm,
  belowskip=2mm,
  showstringspaces=false,
  columns=flexible,
  basicstyle={\small\ttfamily},
  numbers=none,
  numberstyle=\tiny\color{gray},
  keywordstyle=\color{blue},
  commentstyle=\color{dkgreen},
  stringstyle=\color{mauve},
  breaklines=true,
  breakatwhitespace=true,
  tabsize=2
}

\title{%
Design and Analysis of Algorithms\\
\large 6.7 Multipop
}
\author{Daniel Shannon}
\date{May 20th, 2022}

\begin{document}
\begin{titlepage}
\maketitle
\end{titlepage}

\section*{7.7.2}
\begin{quote}
\begin{itemize}
    \item Analyze running time for a sequence of $n$ Push, Pop, Multipop operations
    \item Standard Analysis:
    \begin{itemize}
        \item Push takes $O(1)$
        \item Pop takes $O(1)$
        \item MultiPop takes $O(k)$
        \item Worst case: $n$ MultiPop operations = $O(n^2)$
    \end{itemize}
    \item Calculate amortized running time for each algorithm.
\end{itemize}
\end{quote}
Each push and pop costs 1 credit. In order to be able to pop, we have to have pushed onto the stack first.
So when we push, we pay ahead using a credit for a very cheap operation. Then when we pop we have credit in the bank
from pushing. Averaged over n elements of Multipop, the runtime of MultiPop is $O(k)/k$ since we average over
the whole instruction set, from push to pop.

\subsection*{Solution}
\begin{itemize}
    \item Push: \$1+\$1 to bank
    \item Pop: \$0 to bank, take \$1 from the bank
    \item MultiPop: \$0 to bank, take \$k from bank
\end{itemize}
$$\$2N$$

\end{document}